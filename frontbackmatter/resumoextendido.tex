\chapter*{Zusammenfassung}\addcontentsline{toc}{chapter}{Zusammenfassung}

Blockchiffren bilden ohne Zweifel das Rückgrat unserer heutigen digitalen Kommunikation und werden somit zu Recht als Arbeitstier der Kryptographie bezeichnet.
Während die Effizienz neuer Chiffren stetig steigt, gilt dies nur bedingt für deren Sicherheitsargumente.
Die vorliegende Arbeit beschäftigt sich daher mit zwei Hauptthemen.

In Teil~I untersuchen wir eine neue Notation einer speziellen Kryptanalyse-Technik und geben neue theoretische Einsichten zu dieser.
Außerdem konstruieren wir eine Blockchiffre und zeigen scharfe Schranken für jedes Differential; nach bestem Wissen die erste solche Chiffre.

In Teil~II wenden wir uns algorithmischen Methoden für Design und Analyse von Blockchiffren zu.
Der Hauptbeitrag ist ein Algorithmus um Unterräume durch Chiffren-Runden zu propagieren.
Abschließend diskutieren wir zwei Anwendungen: ein Sicherheitsargument für eine neue Kryptanalyse-Technik und einen Ansatz zur Automatisierung von Key Re\-cov\-ery-Angriffen.
